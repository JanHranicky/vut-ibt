%===============================================================================
% (c) Dominik Harmim


%===============================================================================
\chapter{Experimental Verification Results}
\label{app:expRes}

This appendix illustrates results of experimental verification of
the implemented analyser. The verification process and the results, in
general, are in more detail discussed in Chapter~\ref{chap:exp}.
Section~\ref{sec:expResPhase1} shows experimental results of
\textbf{Phase~1} of the analyser, i.e., the \emph{detection of atomic
sequences}. Section~\ref{sec:expResPhase2} then shows experimental results
of \textbf{Phase~2}, i.e., the \emph{detection of atomicity violations}.

Both sections demonstrate the analysis on programs written in ANSI~C
and assume \emph{PThread} locks and the existence of an initialised
global variable \texttt{lock} of a~type \texttt{pthread\_mutex\_t}.

Moreover, the below experiments are available on the attached memory media,
see Appendix~\ref{app:memMedia}. Instructions how to run these
experiments can be found in Appendix~\ref{app:man}.


\section{Detection of Atomic Sequences}
\label{sec:expResPhase1}

For the verification of the detection of atomic sequences are used
functions defined in Listing~\ref{list:phase1ExpCode}. The result of
this detection, i.e., \emph{sequences of function that should be called
atomically}, is shown in Listing~\ref{list:phase1AtomSeq}.

\begin{lstlisting}[
    style=c, label={list:phase1ExpCode},
    caption={%
        Functions to be analysed for the detection of atomic sequences
    }
]
void f1(void) {}
void f2(void) {}
void f3(void) {}
void f4(void) {}
void f5(void) {}
void ff(void) { f1(); f2(); }

void test1(void)
{
    f1(); f1();

    <@\textcolor{red}{pthread\_mutex\_lock}@>(&lock);
    f1(); f1(); f2();
    <@\textcolor{red}{pthread\_mutex\_unlock}@>(&lock);

    f1(); f1();

    <@\textcolor{red}{pthread\_mutex\_lock}@>(&lock);
    f1(); f3();
    <@\textcolor{red}{pthread\_mutex\_unlock}@>(&lock);

    f1();

    <@\textcolor{red}{pthread\_mutex\_lock}@>(&lock);
    f1(); f3(); f3();
    <@\textcolor{red}{pthread\_mutex\_unlock}@>(&lock);
}

void test2(void)
{
    f1(); f1();

    <@\textcolor{red}{pthread\_mutex\_lock}@>(&lock);
    f1(); f1(); f2();
    <@\textcolor{red}{pthread\_mutex\_unlock}@>(&lock);

    f3(); f3();

    <@\textcolor{red}{pthread\_mutex\_lock}@>(&lock);
    f1(); f4(); f4();
    <@\textcolor{red}{pthread\_mutex\_unlock}@>(&lock);
}

void test_only_lock(void)
{
    <@\textcolor{red}{pthread\_mutex\_lock}@>(&lock);
    f1();
}

void test_only_unlock(void)
{
    f2();
    <@\textcolor{red}{pthread\_mutex\_unlock}@>(&lock);
}

void test_iteration(void)
{
    int c;
    f1(); f2();
    while (c > 0) { f3(); f5(); }

    <@\textcolor{red}{pthread\_mutex\_lock}@>(&lock);
    f1(); f2();
    <@\textcolor{red}{pthread\_mutex\_unlock}@>(&lock);

    <@\textcolor{red}{pthread\_mutex\_lock}@>(&lock);
    while (c > 0) f3();
    <@\textcolor{red}{pthread\_mutex\_unlock}@>(&lock);

    f4();
}

void test_selection(void)
{
    int c;
    f1(); f2();
    if (c > 0) { f3(); f5(); }
    else
    {
        <@\textcolor{red}{pthread\_mutex\_lock}@>(&lock);
        f1();
        <@\textcolor{red}{pthread\_mutex\_unlock}@>(&lock);
    }

    <@\textcolor{red}{pthread\_mutex\_lock}@>(&lock);
    f2();
    if (c > 42) f3();
    else if (c > 0) f4();
    <@\textcolor{red}{pthread\_mutex\_unlock}@>(&lock);

    f4();
}

void test_nested(void)
{
    <@\textcolor{red}{pthread\_mutex\_lock}@>(&lock);
    ff(); f3();
    <@\textcolor{red}{pthread\_mutex\_unlock}@>(&lock);

    <@\textcolor{red}{pthread\_mutex\_lock}@>(&lock);
    f4(); f5(); ff();
    <@\textcolor{red}{pthread\_mutex\_unlock}@>(&lock);
}
\end{lstlisting}

\begin{lstlisting}[
    style=atomSeq, label={list:phase1AtomSeq},
    caption={%
        The result of \textbf{Phase~1} (atomic sequences) of the
        analysis of functions from Listing~\ref{list:phase1ExpCode}
    }
]
f1:<@\textvisiblespace@>
f2:<@\textvisiblespace@>
f3:<@\textvisiblespace@>
f4:<@\textvisiblespace@>
f5:<@\textvisiblespace@>
ff:<@\textvisiblespace@>
test1: (f1 f2) (f1 f3)
test2: (f1 f2) (f1 f4)
test_only_lock: (f1)
test_only_unlock:<@\textvisiblespace@>
test_iteration: (f1 f2) (f3)
test_selection: (f1) (f2) (f2 f3) (f2 f4)
test_nested: (ff f1 f2 f3) (f4 f5 ff f1 f2)
\end{lstlisting}


\section{Detection of Atomicity Violations}
\label{sec:expResPhase2}

For the verification of the detection of atomicity violations are used
functions defined in Listing~\ref{list:phase2ExpCode}. The result of
the detection of atomic sequences is shown in
Listing~\ref{list:phase2AtomSeq}. The result of the detection of atomicity
violations, i.e., \emph{functions that should be called atomically but they
are not}, is stated in Listing~\ref{list:phase2ExpCode} using
\textcolor{greencomments}{comments}.

\begin{lstlisting}[
    style=c, label={list:phase2ExpCode},
    caption={%
        Functions to be analysed for the detection of atomic sequences
        and for the subsequent detection of atomicity violations (detected
        atomicity violations are stated in
        \textcolor{greencomments}{comments})
    }
]
void f1(void) {}
void f2(void) {}
void f3(void) {}
void f4(void) {}
void g(void) {}
void ff(void) { f3(); f1(); f4(); } // (f3 f1) (f1 f4)

void atomic_sequences(void)
{
    <@\textcolor{red}{pthread\_mutex\_lock}@>(&lock);
    f1(); f2(); f3();
    <@\textcolor{red}{pthread\_mutex\_unlock}@>(&lock);

    <@\textcolor{red}{pthread\_mutex\_lock}@>(&lock);
    f4(); f2();
    <@\textcolor{red}{pthread\_mutex\_unlock}@>(&lock);

    <@\textcolor{red}{pthread\_mutex\_lock}@>(&lock);
    f1(); f3();
    <@\textcolor{red}{pthread\_mutex\_unlock}@>(&lock);

    <@\textcolor{red}{pthread\_mutex\_lock}@>(&lock);
    ff(); f3();
    <@\textcolor{red}{pthread\_mutex\_unlock}@>(&lock);
}

void test1(void)
{
    f1(); f2(); g(); // (f1 f2)
    f1(); g(); f2(); g();
    f1(); f1(); f2(); g(); // (f1 f2)
    f1(); f2(); f3(); g(); // (f1 f2) (f2 f3)
    f1(); g(); f2(); g(); f3();
}

void test2(void)
{
    f4(); f2(); g(); // (f4 f2)
    f2(); f4(); g();

    f4();
    <@\textcolor{red}{pthread\_mutex\_lock}@>(&lock);
    <@\textcolor{red}{pthread\_mutex\_unlock}@>(&lock);
    f2();
    g();

    f3(); f4();
}

void test_only_lock(void)
{
    <@\textcolor{red}{pthread\_mutex\_lock}@>(&lock);
    f1(); f2();
}

void test_only_unlock(void)
{
    f1(); f2(); // (f1 f2)
    <@\textcolor{red}{pthread\_mutex\_unlock}@>(&lock);
}

void test_iteration(void)
{
    int c;
    while (c > 0) { f1(); f2(); } // (f1 f2)

    f1();
    while (c > 0) f2(); // (f1 f2)
    f3(); // (f1 f3) (f2 f3)

    for (; c > 0; f1()) f3(); // (f3 f1) (f1 f3)

    <@\textcolor{red}{pthread\_mutex\_lock}@>(&lock);
    f1(); f2();
    while (c > 0) { f2(); f3(); }
    <@\textcolor{red}{pthread\_mutex\_unlock}@>(&lock);
}

void test_selection(void)
{
    int c;
    f1();
    if (c > 0) f2(); // (f1 f2)
    else f3(); // (f1 f3)
    f3(); // (f2 f3)

    g();

    f4();
    if (c > 42) f4();
    else if (c > 0) f2(); // (f4 f2)
    f2(); // (f4 f2)
}

void test_nested(void)
{
    <@\textcolor{red}{pthread\_mutex\_lock}@>(&lock);
    ff();
    <@\textcolor{red}{pthread\_mutex\_unlock}@>(&lock);

    ff(); g(); // (ff f3)
    ff(); f2(); // (ff f3) (f4 f2)
}
\end{lstlisting}

\begin{lstlisting}[
    style=atomSeq, label={list:phase2AtomSeq},
    caption={%
        The result of \textbf{Phase~1} (atomic sequences) of the
        analysis of functions from Listing~\ref{list:phase2ExpCode}
    }
]
f1:<@\textvisiblespace@>
f2:<@\textvisiblespace@>
f3:<@\textvisiblespace@>
f4:<@\textvisiblespace@>
g:<@\textvisiblespace@>
ff:<@\textvisiblespace@>
atomic_sequences: (f1 f2 f3) (f4 f2) (f1 f3) (ff f3 f1 f4)
test1:<@\textvisiblespace@>
test2:<@\textvisiblespace@>
test_only_lock: (f1 f2)
test_only_unlock:<@\textvisiblespace@>
test_iteration: (f1 f2) (f1 f2 f3)
test_selection:<@\textvisiblespace@>
test_nested: (ff f3 f1 f4)
\end{lstlisting}



%===============================================================================
\chapter{Contents of the Attached Memory Media}
\label{app:memMedia}

This appendix lists contents of the attached memory media.

\textbf{The attached memory media particularly contains the following:}
\begin{itemize}
    \item
        \texttt{\textbf{/experiments/}}

        \begin{itemize}
            \item
                The results of the experimental evaluation and experimental
                programs to analyse.
        \end{itemize}

    \item
        \texttt{\textbf{/infer/}}

        \begin{itemize}
            \item
                Source codes of Facebook Infer.

            \item
                The created source files of the implemented analyser are
                located mainly in a~subdirectory \texttt{infer/src/checkers}.

            \item
                In this directory, there are \texttt{.md} files with the
                official instructions on how to install and build this tool.
                But such information is primarily provided in
                Appendix~\ref{app:man}.
        \end{itemize}

    \item
        \texttt{\textbf{/text/}}

        \begin{itemize}
            \item
                \LaTeX\ source codes of this thesis.
            \item
                It can be compiled into the PDF using \texttt{pdflatex} with
                the command \texttt{make}.
        \end{itemize}

    \item
        \texttt{\textbf{xharmi00.pdf}}

        \begin{itemize}
            \item This thesis in the PDF.
        \end{itemize}
\end{itemize}



%===============================================================================
\chapter{Installation and User Manual}
\label{app:man}

This appendix serves as an installation and user manual.

Further, it is assumed that the \emph{working directory} contains all the
files from the attached memory media, see Appendix~\ref{app:memMedia}.


\section*{Installation Manual}

\emph{The whole installation process may be time-consuming.}

At first, it is required to install Facebook Infer's dependencies and then
to compile Facebook Infer. Here are the prerequisites to be able to compile
Facebook Infer on Linux:
\begin{itemize}
    \item opam >= 2.0.0
    \item python 2.7
    \item pkg-config
    \item gcc >= 5.X or clang >= 3.4
    \item autoconf >= 2.63
    \item automake >= 1.11.1
\end{itemize}

Installation of Facebook Infer can be done using the following commands:
\begin{lstlisting}[style=bash]
cd infer
./build-infer.sh clang
sudo make install
\end{lstlisting}
The official installation manual can be found in
\url{https://github.com/harmim/infer/blob/master/INSTALL.md}.

Furthermore, it is needed to build Facebook Infer using these commands:
\begin{lstlisting}[style=bash]
cd infer
make -j BUILD_MODE=default
\end{lstlisting}
The official building manual can be found in
\url{https://github.com/harmim/infer/blob/master/CONTRIBUTING.md}.

Facebook Infer now should be installed and executable by the command
\texttt{infer}. It is installed in \texttt{infer/infer/bin}.


\section*{User Manual}

In general, to analyse a~C/C++ program with Facebook Infer, it can be
done using the following command (for a~single file):
\begin{lstlisting}[style=bash]
infer run -- gcc -c sourc_file.c
\end{lstlisting}
Another option is to analyse the entire project with \texttt{Makefile} using
the following:
\begin{lstlisting}[style=bash]
infer run -- make <target>
\end{lstlisting}
Many other build systems may be used, see
\url{https://fbinfer.com/docs/analyzing-apps-or-projects.html}.

The implemented \textbf{atomicity violations analyser} is deactivated
by default. The analysis have to be executed twice (it has two phases).
Each phase runs with a~different command line option. The first phase
derives sequences of functions that should be executed atomically into
the file \texttt{infer-atomicity-out/atomic-sequences}. The second phase
then detects atomicity violations according to this file. So, the analysis
can be triggered using the following commands:
\begin{lstlisting}[style=bash]
infer run --atomic-sequences-only -- gcc -c sourc_file.c
infer run --atomicity-violations-only -- gcc -c sourc_file.c
\end{lstlisting}
Or it can be triggered along with other analyses that Facebook Infer provides:
\begin{lstlisting}[style=bash]
infer run --atomic-sequences -- gcc -c sourc_file.c
infer run --atomicity-violations -- gcc -c sourc_file.c
\end{lstlisting}


%===============================================================================
